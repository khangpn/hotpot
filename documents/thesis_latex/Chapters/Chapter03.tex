% Chapter 3

\chapter{HotPot Project} % Chapter title

\label{ch:hopot_project} % For referencing the chapter elsewhere, use \autoref{ch:hotpot_project}

%----------------------------------------------------------------------------------------

In \autoref{ch:background}, we learned that MLS is a very practical solution for multi-level access control.
It is very suitable to PMT because of the essence of PMT is sharing and collaborating among groups of people of different concerns and priorities.
As introduced in \autoref{ch:introduction:use_cases}, \myProject is a PMT implementing BLP model to ensure data security by forcing subjects to follow a set of predefined rules in order to gain access permission to objects.
In this chapter, I am going to explain in details the features of the project; how is it different from the other PMT; how does it approach the security demands.
By that way, we can understand precisely the goal of the project.

%----------------------------------------------------------------------------------------

\section{Concepts}
\label{ch:hopot_project:concepts}

By the time of writing this thesis, in the market, there are already plenty of PMTs with a huge number of features.
Some of popular names are Redmine, Workfront, FreshDesk, Genius Project etc...
Every tools has their own key features.
Go through their features list, we can see that most of features mainly focus on management and analyzing purposes.
They aim to provide customers most convenient tool with rich features and best user experience.
In term of security, most of the projects support SSL data encryption and apply vairous techniques to ensure data integrity.
Those traits can protect data on communicating progress, or keep it away from external exploits.
However, in order to protect internal assets from being accessed by authenticated users, most of them only support \emph{Conventional Access Permissions} (CAP) which allow read/write permissions on an asset.
In fact, many of data leaking cases were reported done by authenticated but unauthorized users.
MLS perfectly matches into this scenario that it enforces all internal subjects to follow accessing policies, regardless of their priorities or the object's CAP.
In another word, every objects in the system will served exactly to groups of subjects who satisfy their admission requirements.
As the result, MLS is a eminently strong solution for lacking of \emph{Mandatory Access Control} (MAC).
And that is also the main goal of \myProject that tries to implement BLP model to PMT.

\myProject is a PMT with essential features of managing projects.
At this stage, because of the purpose of this thesis, I only primarily focus on security practices of BLP in PMT.
And the management features are only enough to illustrate a general working PMT.

%------------------------------------------------

\subsection{Project Management Features}
\label{ch:hopot_project:concepts:pm_features} 

PMT is a platform in which users can plan their projects, create and track project tasks, project progress, team members' assignments etc...
In the market, many PMTs have outstanding user experience features.
They consist of performance analyzing and evaluation tool, both personally and organizationally.
Some even embed discussion forum, subversion control monitoring etc...
However in \myProject, I only focus on some essential features of a PMT in order to demonstrate the security features such as:

\begin{itemize}
\item Create and manage (edit, delete) accounts. There are two types of account: normal account and admin, we will go into details about these types in the next \autoref{ch:hopot_project:project_components:account}
\begin{itemize} 
\item Editting account's personal information (\eg full name, email, address, contact etc...). 
After logging in, users can update their own information.
\end{itemize}

\item Create and manage (edit, delete) projects.
\begin{itemize} 
\item Assigning members to projects. 
\item Setting project members' security labels (\eg \emph{security level}, \emph{roles}).
\end{itemize}
\item List all projects.
\item List project's members.

\item Create and manage (edit, delete) articles.
\begin{itemize} 
\item Creating directory.
\item Setting articles and directories' security labels (\eg \emph{security level}, \emph{roles}).
\end{itemize}
\item List project's articles: the result of the articles list will be different from users. Because each account will have different set of security labels.

\item Create and manage (edit, delete) tickets.
\begin{itemize} 
\item Tickets assigning.
\item Setting tickets' security labels (\eg \emph{security level}, \emph{roles}).
\end{itemize}
\item List project's tickets: this list is similar to articles list.

\item Create and manage (edit, delete) security components. \footnote{They are the core of the security policies}
\begin{itemize} 
\item Roles.
\item Security Levels.
\end{itemize}.

\end{itemize}

In the next \autoref{ch:hopot_project:concepts:security_features}, we will learn about the projects' \emph{security policies}.
It will help us to answer questions related to management features such as who can perform these functions? How security policies will affect their performance? etc...

%------------------------------------------------

\subsection{Security Features}
\label{ch:hopot_project:concepts:security_features}

This chapter focus on answering one big question \emph{How do we handle security?}
It will explain in details the \emph{security policies} of the project.
So that, We can understand how users can use this system? What rules they have to follow? And how the restriction can protect data by users with multi-level security?

Because of using MLS, first of all, we have to list all of the system's components, then categories them as either \emph{subject} or \emph{object}.
\emph{Subjects} are components which actively interact in the system \eg\ processes, users etc\dots
So, in our project, they are \emph{accounts} (or users, team members\dots) and another special type of account is \emph{administrators} (or admins for short).
On the other hand, \emph{objects} are passive components which are \emph{articles}, \emph{tickets}, and \emph{projects} in our case.
Rather than that, there are three elements that form the security policies for subjects and objects. They are \emph{security levels}, \emph{roles}, and \emph{read/write permissions}.
Every subjects and objects will have one security level label and at least one role. Besides that, all the objects will have one more CAP feature, so they can restrict the subjects' behaviors such as read or write.
In the system, there can be many projects developed at the same time.
Everyone, every subjects, can have different roles, and priorities in different projects.
Consequently, in order to ensure the system consistency, every subject in different projects will have different \emph{project profiles}.
Each profile consists of a different set of security labels, corresponding to the subject's characters in the project. %draw sth to illustrate this

In \myProject, there are main actions a subject can perform on an object: \emph{create, view, list, edit, delete}.
Some of them such as \emph{view, list} are considered as reading behavior, so that it requires \emph{read permission} on the objects for the subjects to perform.
In a different manner, \emph{create, edit and delete} require \emph{write permission} on the object for the subject to perform.
\marginpar{\emph{Edit} action need both read and write permission.}
However, considering about \emph{edit} action, we can see that in order for a subject to edit an object's contents, he must be able to read it first.
So \emph{edit} is a special case where it requires the object's both \emph{read} and \emph{write} permission.
\marginpar{\emph{Delete} action can only be done by the object's owner.}
Also \emph{delete} is also a sensitive action.
It is a \emph{write action}.
Yet, in term of data integrity, \emph{delete} action should only be executed by the object's owner \ie subject which created it.
Let us consider this scenario, a secret article should be created to report a fault of a team member to the managers.
If a manager with secrete label does not want that report to be available, he can delete it.
Nevertheless, \emph{create} is also a distinctive action.
It also depends on other logical rules for the subject to perform.
For example, only team member can create articles and tickets of that project.
We will discuss about these kinds of special actions and logical rules in the upcoming \autoref{ch:hopot_project:project_components}.
According to BLP security policies explained in \autoref{ch:background:bell}, every subjects must own a set of security labels which full-fills the security policies again the target object's.
So in order to perform \emph{read action}, the subject's security level label should be higher or equals to the object's;
the subject's set of role labels must be a subset of the object's;
and the object must be readable \ie read permission is on.
On the other hand, the \emph{write action} requires the subject's security level label is lower or equals to the object's;
the subject's set of role labels must be a subset of the object's too;
and the object is set to writable mode.

% NOTE: add illustraion for the security decision.

On developing the project, I sometimes had to due with trade-off of strong security and convenient user experience.
Solving object's owner's security policies is one of the trade-off I had to make.
Let us consider a scenario.
A subject, with secret security label, creates an object \eg article with top secret label. 
After that, he needs to \emph{edit} the article, \eg add more information, but the security policies will prevent him from doing that because, as we explained above, he need to have the same security level with the artilce to \emph{edit} it.
This will be very inconvenient in case he has to edit the article many times.
On the contrary, if we allow the owner to read his objects regardless of their security level, it may be vulnerable to data leak.
Since, if a top secret subject decides to edit the object to add some sensitive data into it, the owner can still read them.
\marginpar{\emph{Trade-off}: a subject have all permissions on his own objects.}
However, thinking about the user experience and because the project aims to serve civil organizations of medium scale, I think it is better to let owners have all permissions on their own objects.
It also helps to solve the problem of \emph{delete} action that we discussed before.
\marginpar{\emph{Future plan}: logging system is needed to track all actions in the system, especially these kind of compromising actions that need the subjects' awareness.}
Because this is a compromise, so that we need a good practice to fix it.
It could be a system unrelated rule, that high security level subjects should not edit objects of lower security level subject, instead he should create a copy of it then add his additional information into the copy, unless he is only leaving comments on the object.

Furthermore, besides the BLP model policies, some logical rules are applied for some actions in the system.
These rules also have to follow the predefined security policies.
They are just like another layer of the access control to the actions.
For example, creating new subjects (accounts); managing relations among objects; creating new objects related to another objects etc\dots
Those logical rules will be discussed in the next \autoref{ch:hopot_project:project_components}.
%----------------------------------------------------------------------------------------

\section{Project Components and Use Cases}
\label{ch:hopot_project:project_components}

How we build the project? what we focus on? What security principle should the follow?

This chapter will explain about security behaviors of every components of the project.
We will learn about all the main features of the system, as well as how they are expected to behave against security policies.
Here, we also discuss about the logical polices that could be considered another layer of the access control.
Finally, we will explain most of the use cases of the project.
These use cases will be demonstrated illustratively later in \autoref{ch:result:user_guide}.

%------------------------------------------------

\subsection{Account and Administrator}
\label{ch:hopot_project:project_components:account}

Account is the main subject in our system.
There are two types of account: \emph{user account} and \emph{admin account}.
These two types of account have the same properties in database, however, due to their responsibilities ,they will have different behavior in the system.

A user account is the account type for most of everyone in the organization.
It consists of enough information of an employee \eg username, password, contact \dots
Below are lists of all actions the account can perform on other objects in the system. These actions will be explained in more detailed when will discuss about the other components in this chapter.

\afterpage{
%account
\begin{table}[!htbp]
\myfloatalign
\begin{tabularx}{\textwidth}{lX} 
\toprule
\tableheadline{Actions} & \tableheadline{Description}\\ 
\midrule
\emph{Create} & 
ONLY administrator can create new account.\\
\midrule
\emph{View} & 
Every account can view the other's details. 
This is because of the simplicity and convenience of the project.
In fact, there is no vital data in this area that can be exploited.\\
\midrule
\emph{Edit} & 
ONLY the \emph{account's owner} can update his own account's details. \\
\midrule
\emph{Delete} & 
Administrator and the account owner can delete the account.\\
\bottomrule
\end{tabularx}
\caption[Account's actions on account details.]{Account's actions on account details.}  
\label{tab:account_on_detail}
\end{table}

%project
\begin{table}[!htbp]
\myfloatalign
\begin{tabularx}{\textwidth}{lX} 
\toprule
\tableheadline{Actions} & \tableheadline{Description}\\ 
\midrule
\emph{Create} & 
Every account can create a new project. The creator account, then, will be set as the project's owner, and added to the project as a project member.\\
\midrule
\emph{View} & 
Every account can view the project information. 
However, the visitor cannot see the list of articles and tickets of the project.\\
\midrule
\emph{Edit} & 
ONLY the \emph{project owner} can update the project's information. \\
\midrule
\emph{Add member} & 
ONLY the \emph{project owner} can add the other accounts to his project. \footnote{Every accounts added to a project will have one \emph{project profile} where store all of the account's security labels in that project} \\
\midrule
\emph{Edit member permission} & 
ONLY administrators can perform this action. \\
\midrule
\emph{Delete} & 
ONLY the \emph{project's owner} can delete the project. \\
\bottomrule
\end{tabularx}
\caption[Account's actions on projects.]{Account's actions on projects.}  
\label{tab:account_on_project}
\end{table}

%article
\begin{table}[!htbp]
\myfloatalign
\begin{tabularx}{\textwidth}{lX} 
\toprule
\tableheadline{Actions} & \tableheadline{Description}\\ 
\midrule
\emph{Create} & 
Every project member can create new articles of that project. The creator account, then, will be set as the article's owner. This action must follow the security policies in \autoref{ch:hopot_project:concepts:security_features}.\\
\midrule
\emph{List} & 
Every project member can see the list of project's articles.
The list only shows articles related to the member's security labels, that follow the security policies in \autoref{ch:hopot_project:concepts:security_features}.\\
\midrule
\emph{View} & 
Every project member can view the project's articles.
This action must follow the security policies in \autoref{ch:hopot_project:concepts:security_features}.\\
\midrule
\emph{Edit} & 
Every project member can edit the project's articles. 
This action must follow the security policies in \autoref{ch:hopot_project:concepts:security_features}.\\
\midrule
\emph{Add article to directory} & 
Every project member can add articles to a the project's article directory 
\footnote{\emph{Article} and \emph{directory} relation will be discuss in \autoref{ch:hopot_project:article}}.
Adding an article to a directory is considered as editing the directory so that this action must follow the security policies in \autoref{ch:hopot_project:concepts:security_features} against on directory.\\
\midrule
\emph{Delete} & 
ONLY the \emph{article's owner} can delete it. \\
\bottomrule
\end{tabularx}
\caption[Account's actions on articles.]{Account's actions on articles.}  
\label{tab:account_on_article}
\end{table}

%ticket
\begin{table}[!htbp]
\myfloatalign
\begin{tabularx}{\textwidth}{lX} 
\toprule
\tableheadline{Actions} & \tableheadline{Description}\\ 
\midrule
\emph{Create} & 
Every project member can create new tickets of that project. The creator account, then, will be set as the ticket's owner. This action must follow the security policies in \autoref{ch:hopot_project:concepts:security_features}.\\
\midrule
\emph{List} & 
Every project member can see the list of project's tickets.
The list only shows tickets related to the member's security labels, that follow the security policies in \autoref{ch:hopot_project:concepts:security_features}.\\
\midrule
\emph{View} & 
Every project member can view the project's tickets.
This action must follow the security policies in \autoref{ch:hopot_project:concepts:security_features}.\\
\midrule
\emph{Edit} & 
Every project member can edit the project's tickets. 
This action must follow the security policies in \autoref{ch:hopot_project:concepts:security_features}.\\
\midrule
\emph{Assign ticket} & 
This action is similar to \emph{edit} action.\\
\midrule
\emph{Delete} & 
ONLY the \emph{ticket's owner} can delete it. \\
\bottomrule

\end{tabularx}
\caption[Account's actions on tickets.]{Account's actions on tickets.}  
\label{tab:account_on_ticket}
\end{table}
}

Besides that, there are some actions on security related components like \emph{security labels}, \emph{role} that can only be performed by \emph{administrator accounts}. Administrators have all permissions on these special components. Besides that, they are the only ones who can edit other accounts' security labels.
\cleardoublepage %fix the tables' appearance problems

%------------------------------------------------

\subsection{Project}
\label{ch:hopot_project:project_components:project}

Project is not actually neither subject nor object of the system.
In the manner of management, it is just a grouping element to group related objects \eg accounts, articles and tickets together.

According to \autoref{tab:account_on_detail}, all accounts can create new project.
Then, the project's owner can add other accounts to the project as team members.
Each of the team members will have one \emph{project profile} which store all needed information for the system to apply the security policies on the member's actions \eg \emph{security level}, roles.
By that design, every accounts will have different priorities in different projects, which provide a flexible management in the organization.
Moreover, only the owner can edit and delete the project.

Each projects has its own lists of articles and tickets.
Only the project's team members can access them \eg create, edit, list, view delete.

%------------------------------------------------

\subsection{Security Level (Clearance)}
\label{ch:hopot_project:project_components:security_level}

\emph{Security Level} is one of the security components.
It represent the \emph{clearance} property of objects and subjects.
Only an administrator can access to perfom actions on this component.
The \emph{security level label} has a \emph{level} property to indicate its priority.
The higher number the level, the more important the owner is.
For example, \emph{Secret} label has the \emph{level} value of 3, and \emph{Top Secret} has the value of 4.
The name of the security level label as well as its priority level can completely be decided by the system's administrators.
In another word, the administrators can modify the system's appearance as his will. 
The security level label will relate to an account via his project profiles. Every account's project profiles will have a different one.

%------------------------------------------------

\subsection{Role (Classification)}
\label{ch:hopot_project:role}

Like \emph {Security Level}, \emph{Role} is also a security component of the system.
It indicate subjects, objects' categories \ie \emph{classification}.
As we discussed in previous \autoref{ch:background:bell}, one subject or object can have various role labels.
And a subject's roles have to be a subset of the object's in order for the subject to access it.
Similar to \emph{Security Level}, \emph{Roles} also relate to an account via hist project profiles. 
In another word, one subject may have different roles in different project.
Indeed, it's a great common scenario in organizations.


%------------------------------------------------

\subsection{Article and Directory}
\label{ch:hopot_project:article}

\emph{Article} is one of two main objects of the system.
Article could be any kinds of document.
It could be a report, technical notes, project related documentation etc\dots
Every article belongs to a particular project.
Any team member in the project can create new articles.
While every other team members have to full-fills the security policies described in \autoref{ch:background:bell} in order to access an article, the article's owner has all permission on it.
And, only the owner can delete his articles.

In some cases, many articles, which relate to each other and share similar security policies, should be grouped together.
It helps us to manage articles better, and provides a flexible hierachy for sharing articles.
In that manner, \emph{Directory} is designed to deliver that feature.
In fact, \emph{directory} is an \emph{article}.
It means that it has all article's properties, and follows all article's security policies, except that a directory can have the other articles as its children.
The children articles created inside a directory should have equal or higher \emph{security level} than the directory.
So that, team members can \emph{create} new article inside the directory without getting restricted by security polices.
Rather than that, the children should have a set of roles which is a subset of the directory's roles.
It will generate a common sense for team members when they want to find some articles inside a directory, they can check the directory's roles to know the children's categories.
Finally, because \emph{directory} is an \emph{article}, so it also have read and write permission on it.

%------------------------------------------------

\subsection{Ticket}
\label{ch:hopot_project:ticket}

Just like \emph{Article}, \emph{Ticket} is also a kind of subject of the system.
It also need to follow security policies defined in \autoref{ch:background:bell}, and it belongs to projects too.
Every tickets will have an owner (or creator) and an assignee property.
Assigning an assignee to a ticket is also considered as an \emph{edit action}, so that the subject who performs it also need full-fills the edit policies.
Moreover, the assignee's security policies should satisfy the ticket's.
