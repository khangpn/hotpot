% Chapter 3

\chapter{HotPot Project} % Chapter title

\label{ch:hopot_project} % For referencing the chapter elsewhere, use \autoref{ch:hotpot_project}

%----------------------------------------------------------------------------------------

In \autoref{ch:background}, we learned that MLS is a very practical solution for multi-level access control.
It is very suitable to PMT because of the essence of PMT is sharing and collaborating among groups of people of different concerns and priorities.
As introduced in \autoref{ch:introduction:use_cases}, \myProject is a PMT implementing BLP model to ensure data security by forcing subjects to follow a set of predefined rules in order to gain access permission to objects.
In this chapter, I am going to explain in details the features of the project; how is it different from the other PMT; how does it approach the security demands.
By that way, we can understand precisely the goal of the project.

%----------------------------------------------------------------------------------------

\section{Concepts}
\label{ch:hopot_project:concepts}

By the time of writing this thesis, in the market, there are already plenty of PMTs with a huge number of features.
Some of popular names are Redmine, Workfront, FreshDesk, Genius Project etc...
Every tools has their own key features.
Go through their features list, we can see that most of features mainly focus on management and analyzing purposes.
They aim to provide customers most convenient tool with rich features and best user experience.
In term of security, most of the projects support SSL data encryption and apply vairous techniques to ensure data integrity.
Those traits can protect data on communicating progress, or keep it away from external exploits.
However, in order to protect internal assets from being accessed by authenticated users, most of them only support CAP which allow read/write permissions on an asset.
In fact, many of data leaking cases were reported done by authenticated but unauthorized users.
MLS perfectly matches into this scenario that it enforces all internal subjects to follow accessing policies, regardless of their priorities or the object's CAP.
In another word, every objects in the system will served exactly to groups of subjects who satisfy their admission requirements.
As the result, MLS is a eminently strong solution for lacking of MAC.
And that is also the main goal of \myProject that tries to implement BLP model to PMT.

\myProject is a PMT with essential features of managing projects.
At this stage, because of the purpose of this thesis, I only primarily focus on security practices of BLP in PMT.
And the management features are only enough to illustrate a general working PMT.

%------------------------------------------------

\subsection{Project Management Features}
\label{ch:hopot_project:concepts:pm_features} 

PMT is a platform in which users can plan their projects, create and track project tasks, project progress, team members' assignments etc...
In the market, many PMTs have outstanding user experience features.
They consist of performance analyzing and evaluation tool, both personally and organizationally.
Some even embed discussion forum, subversion control monitoring etc...
However in \myProject, I only focus on some essential features of a PMT in order to demonstrate the security features such as:

\begin{itemize}
\item Create and manage (edit, delete) accounts. There are two types of account: normal account and admin, we will go into details about these types in the next \autoref{ch:hopot_project:project_components:account}
\begin{itemize} 
\item Editting account's personal information (\eg full name, email, address, contact etc...). 
After logging in, users can update their own information.
\end{itemize}

\item Create and manage (edit, delete) projects.
\begin{itemize} 
\item Assigning members to projects. 
\item Setting project members' security labels (\eg \emph{security level}, \emph{roles}).
\end{itemize}

\item Create and manage (edit, delete) articles.
\begin{itemize} 
\item Creating directory.
\item Setting articles and directories' security labels (\eg \emph{security level}, \emph{roles}).
\end{itemize}

\item Create and manage (edit, delete) tickets.
\begin{itemize} 
\item Tickets assigning.
\item Setting tickets' security labels (\eg \emph{security level}, \emph{roles}).
\end{itemize}

\item Create and manage (edit, delete) security components. \footnote{They are the core of the security policies}
\begin{itemize} 
\item Roles.
\item Security Levels.
\end{itemize}

\end{itemize}

In the next \autoref{ch:hopot_project:concepts:security_features}, we will learn about the projects' \emph{security policies}.
It will help us to answer questions related to management features such as who can perform these functions? How security policies will affect their performance? etc...

%------------------------------------------------

\subsection{Security Features}
\label{ch:hopot_project:concepts:security_features}

This chapter focus on answering one big question \emph{How do we handle security?}
It will explain in details the \emph{security policies} of the project.
So that, We can understand how users can use this system? What rules they have to follow? And how the restriction can protect data by users with multi-level security?

Because of using MLS, first of all, we have to list all of the system's components, then categories them as either \emph{subject} or \emph{object}.
\emph{Subjects} are components which actively interact in the system \eg\ processes, users etc\dots
So, in our project, they are \emph{accounts} (or users, team members\dots) and another special type of account is \emph{administrators} (or admins for short).
On the other hand, \emph{objects} are passive components which are \emph{articles}, \emph{tickets}, and \emph{projects} in our case.

%----------------------------------------------------------------------------------------

\section{Project Components}
\label{ch:hopot_project:project_components}

How we build the project? what we focus on? What security principle should the follow?

%------------------------------------------------

\subsection{Account and Admin}
\label{ch:hopot_project:project_components:account}

Explain account

%------------------------------------------------

\subsection{Project}
\label{ch:hopot_project:project_components:project}

Content

%------------------------------------------------

\subsection{Security Level (Clearance)}
\label{ch:hopot_project:project_components:security_level}

Content

%------------------------------------------------

\subsection{Role (Classification)}
\label{ch:hopot_project:role}

Content

%------------------------------------------------

\subsection{Article and Directory}
\label{ch:hopot_project:article}

Content

%------------------------------------------------

\subsection{Ticket}
\label{ch:hopot_project:ticket}

Content
