% Chapter 6

\chapter{Future plan} % Chapter title

\label{ch:future_plan} % For referencing the chapter elsewhere, use \autoref{ch:name} 

%----------------------------------------------------------------------------------------
Although this project just serves the purpose of writing the thesis, it, in my own opinion, is very potential to become a professional \emph{Project Management Tool} (PMT).
However, in the manner of a PMT, there are still a great number of features that could be implemented in the future.
As the purpose of the project is to demonstrate the MLS features, the project's PM features are only enough to cover essential functions to perform managing tasks.
Most of the future features can focus on improving user experiences, performance, and data analyzing.

In order to improve user experiences and performance, the \emph{user personal page} should include more functions such as:
\begin{description}
\item[The user's tickets list] only show tickets which are assigned to the user and sorted as priority and deadline.
\item[Timer] every tasks should have a timer to track the consumed time it takes to finish the task.
It could also be used to track the user's working time in a day.
\item[Ticket diffuser] looking up the tickets is sometimes stressful, especially when there are so many of them.
The diffuser will return one ticket at the time based on its priority, due time, and project's due time\dots
\end{description}

Secondly, \emph{ticket} is also a very important element of the system.
So far, it still lacks of some important functions which can be added in the future:
\begin{description}
\item[Ticket status] the ticket should have various status \eg new, in progress, done, feedback\dots
\item[Sub tickets] like articles in a directory, a ticket may describe a task which requires additional tasks to be done, or it can be divided into smaller tasks to distribute to different team members.
\end{description}

In addition, at the moment, \emph{article} only consists of simple information and features, some of improvement could be applied are:
\begin{description}
\item[Rich content editor] \emph{article}'s contents may various contents like text with styles, images, attachments etc\dots
\item[Comments] in some case (\eg security policies), the viewer can not edit the article's contents, he can leave the comments on the article.
This can also be used to discuss about a particular article.
\end{description}

Besides that, analyzing tools are also crucial elements in PMTs.
It helps managers have an overview about projects in the organization, as well as checking employees' productivity.
Some of essential analyzing functions are:
\begin{description}
\item[Project's tasks] Every project has a time line of its progress.
The time line will show the number of tasks which are either new, in progress, or done.
It can be filtered as daily, weekly, or monthly.
\emph[Project's features] a time line show the completed features.
\end{description}
Similar to projects, users should have analyzing tools to sum up their performance in a certain time \eg \emph{tickets report}, \emph{working time report}\dots

Furthermore, some other system's tools could also be useful, such as:
\begin{description}
\item[Event notification] system's events such as newly assigned tickets, messages from other users etc\dots can be notified to the user immediately.
\item[Git laboratory] display the project's source code repository illustratively.
\end{description}

On the other hand, in term of security, the requests between clients and servers should be encrypted using \emph{SSL}.

Finally, there are still some tweaks needed in the system.
The user interface needs to redesign in order to improve user experience.
The error messages display is not consistent, it should have a more user-friendly design, and give meaningful instructions while not give away the internal system's information.

These features will enhance the system's performance, make it handy, hence increase its user experience.
After the boost, this project could completely become a leading PMT in the market.
Finally, we can shift this project to mobile application. 
The application will call the same \emph{API} except that it wouldn't return HTML but the corresponding pieces of data.
%----------------------------------------------------------------------------------------
