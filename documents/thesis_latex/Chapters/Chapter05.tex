% Chapter 5

\chapter{Project Outcome} % Chapter title

\label{ch:outcome} 
% For referencing the chapter elsewhere, use \autoref{ch:outcome} 

%----------------------------------------------------------------------------------------

The project ended up with a web-based project management system.
The system consists of basic managing features which cover most common scenarios in PMT.
Besides that, the system also follows a set of MLS policies described in \autoref{ch:background:bell}.
This chapter includes an installation guide to setup the project from fresh; 
the system's user guide which describes its features and their usage demonstrations.
And at the end, I will discuss about my learning from the project's developing progress.

%----------------------------------------------------------------------------------------

\section{Installation Guide}
\label{ch:result:installation_guide}

\autoref{ch:implementation:technical_information} listed the technical requirements for this project.
In this guide, we will learn about the project's setup from scratch, step by step.

Assume that you already have a machine running CentOS 7 with these packages have already been installed: \emph{git, postgresql, nodejs with npm}.
I am using \emph{Digital Ocean} as VPS provider.
I create a \emph{Digital Ocean's droplet} with this specification: \emph{512MB Ram 20GB SSD Disk CentOS 7.1 x64}.
A postgresql database should be created and named as \emph{hotpot} whose owner is \emph{hotpot} and the password at your own choice.

First of all we need to clone the project's source code from GitHub using the command.
\begin{lstlisting}[breaklines=false,frame=lt]
$ git clone git@github.com:khangpn/hotpot.git
\end{lstlisting}

Next, we need to install all the project's dependencies by going into the project's root directory and execute \emph{npm} install command
\begin{lstlisting}[breaklines=false,frame=lt]
$/hotpot/ npm install
\end{lstlisting}
This command may take a while to install all required packages.

After checking the database accessibility, we need to configure its connection in \emph{config\/database.js}. 
Open that file by your favorite text editor, my choice is \emph{vim}. 
\begin{lstlisting}[breaklines=false,frame=lt]
$/hotpot/ vim config/database.js
\end{lstlisting}

Below is an example of the database configuration, if you named the database, its user and password are \emph{hotpot} and the database is running on local machine, then your configuration file should be similar to this.
\begin{lstlisting}[breaklines=false,frame=lt]
var settings = {                                                                                                                                                     
  development: { 
    database : "hotpot",
    username : "hotpot",
    password : "hotpot",
    options  : { 
      dialect : "postgresql",
      host     : "127.0.0.1"   
    }
  },
  production: { 
    database : "hotpot",
    username : "hotpot",
    password : "hotpot",
    options  : { 
      dialect : "postgresql",  
      host     : "127.0.0.1"   
    }
  }
};
  
module.exports = settings;
\end{lstlisting}

Next, lets open file \emph{setup} and change its \emph{NODE\_ENV} to \emph{development} or \emph{production} depending on your purpose.
\emph{Development} environemnt will print out the system's logs on our server and return errors' full stack trace to client, while  \emph{production} will omit it.
\begin{lstlisting}[breaklines=false,frame=lt]
// file hotpot/setup
NODE_ENV=development DEBUG=hotpot node bin/db-integration.js
\end{lstlisting}

Also edit file \emph{start} to change its \emph{NODE\_ENV}
\begin{lstlisting}[breaklines=false,frame=lt]
// file hotpot/start
NODE_ENV=development DEBUG=hotpot npm start
\end{lstlisting}
We can also add \emph{PORT} variable to the script to specify the project's running port. By default, it runs on port \emph{3000}
\begin{lstlisting}[breaklines=false,frame=lt]
// file hotpot/start
NODE_ENV=development PORT=4000 DEBUG=hotpot npm start
\end{lstlisting}

Next, we have to initialize the project's database schema using the \emph{setup} script.
\begin{lstlisting}[breaklines=false,frame=lt]
$/hotpot/ ./setup
\end{lstlisting}
This setup may take a while. And finally we can start the project using the \emph{start} script.
\begin{lstlisting}[breaklines=false,frame=lt]
$/hotpot/ ./start
\end{lstlisting}

%----------------------------------------------------------------------------------------

\section{User Guide}
\label{ch:result:user_guide}

used technologies. system requirement, test cases and their results to show how system works.

%----------------------------------------------------------------------------------------
\section{Learnings}

what I learn from the project
