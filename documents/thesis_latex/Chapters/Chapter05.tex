% Chapter 5

\chapter{Project Outcome} % Chapter title

\label{ch:outcome} 
% For referencing the chapter elsewhere, use \autoref{ch:outcome} 

%----------------------------------------------------------------------------------------

The project ended up with a web-based project management system.
The system consists of basic managing features which cover most common scenarios in PMT.
Besides that, the system also follows a set of MLS policies described in \autoref{ch:background:bell}.
The \emph{User Guide}, including \emph{Installation Guide} and \emph{Manual}, is one deliverable of the projecet and it is presented in \autoref{ch:appendix-a}.
Moreover, \autoref{ch:appendix-b} represents the project repository, its structure and contents.

In \autoref{ch:outcome:learnings} , I will discuss about my learning from the project's developing progress.


%----------------------------------------------------------------------------------------
\section{Learnings}
\label{ch:outcome:learnings} 

As working on the project, I have learned many new things.
First of all, technically, this is my second \emph{Nodejs} project.
So in this project, I deliberately chose it for study purpose.
After the project, I had an improving about \emph{Nodejs} and one of its most popular framework \emph{Expressjs}.
All of my experiences and learning about it are noted in \emph{diary} directory on GitHub repository which is listed in \autoref{ch:appendix:repository_structure}.
Those diaries are my daily reports of what I have done, the problems occurred on that day, and their solutions if available\dots

About programming experience, there is one advise which may be very useful when we have to due with unfamiliar issues of new technologies: always spend time for reading the technology's documents.
\marginpar{``Give me six hours to chop down a tree and I will spend the first four sharpening the axe'' -- Abraham Lincoln (1809-1865) }
Obviously, with the strong supports of huge technical communities these days, we can easily finish our jobs by searching for available solutions.
However, it does not help us to master the technology on our own \ie we don't understand it enough to handle other problems in the future.
Reading documentation will profit us in understand the technology's principles, from that we can study best practices in using it.
Consequently, it will improve our performance in using it.
On the other hand, spending time on learning a technology properly, we will also learn its owner's experience on programming and problems solving skills.
In another word, we can inherit his programming mindsets, design patterns which could be profitable in our future projects.
My suggestion is to put technology studying as a task in the project's tasks list, and assign a time span to it as an usual one.
The time span should be long enough for reading through the technology's documents, and not too long or it will become time consuming.
It took me two days to learn about \emph{Expressjs} enough for me to start working with it initially.
And it takes the same amount of time for me to studying using some crucial libraries such as \emph{Sequelizejs}, \emph{Promise}\dots
And on the working progress, I can learn more about their good practices.

Besides that, studying about MLS is a huge advantage to me.
Security is always a hot and essential topic, MLS is one of it.
Through the project, I learned about the MLS policies, and how they could be implemented in a system.
We have to design a blueprint of the system based on selected MLS model.
The blueprint consists of rules that \emph{subjects} and \emph{objects} of the system must be devoted to.
In the progress, we need to review it multiple times in multiple points of view to ensure that there is no exception.
Rather than that, again, there is no \emph{one master solution for all problems}, sometimes we have to make compromises.
In another word, for example, we have to make trade-off between performance and security.
The high secured system may be complicated and inflexible; however, flexible system with many \emph{trusted subjects/objects} could pose security leaks in the future.
So that, whenever we have to make a compromise, we have to acknowledge and keep in mind (or better, in note) its consequences to avoid in future work.

Planing is also an extensive skill.
Although, this project is my personal project, I learned a lot of time management.
In my own opinion, at the first try, we should assign a task with a small extension of time than we have evaluated.
We should not give a too tight time slot for very first tasks, there will be usually some modifications in features list and project requirements at the first stage of the project; and making early decisions very likely leads to changes.
As the progress goes on, the project is also getting stable.
The number of sudden changes or making decision will decrease remarkably.
Consequently, at this time, we can shorten the addition time extension, or just omit it.

