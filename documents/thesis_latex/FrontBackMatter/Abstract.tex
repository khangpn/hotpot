% Abstract

\pdfbookmark[1]{Abstract}{Abstract} % Bookmark name visible in a PDF viewer

\begin{flushleft}{\slshape    
UNIVERSITY OF TURKU \\
Department of Information Technology \\
\medskip
PHUC-KHANG NGUYEN: Multilevel Security in Project Management \\
\medskip
Master thesis, 104 p., 1 app. \\
\medskip
2016
} 
\end{flushleft}

%----------------------------------------------------------------------------------------

\begingroup
\let\clearpage\relax
\let\cleardoublepage\relax
\let\cleardoublepage\relax

\chapter*{Abstract} % Abstract name

In IT industry, in term of \emph{project management} (PM), there are various choices of solution for management tools \eg Redmine, FreshDesk, ProjectLibre, LibrePlan, OpenProject, ProjectOpen etc...
They can cover most of organization's needs, including security.
However, at the time of writing this document, none of those solutions covers protecting information in a multi level environment, \ie many roles, \eg developers, designer, manager, CEO etc..., working on a same system, which is a typical situation in most of companies. 

A \emph{Multi-level Security} (MLS) system has many users with different clearance levels using a same system (or machine) without exposing their data to unreasonable users.
%For example, a user with \emph{roles} of programmer is very unlikely to need to know about designers' documents;
%a programmer with \emph{security level} 'secret' should not be able to read 'top secret' documents\dots
%These predefined accessing rules help to prevent data among the system to be mishandled, or leaked to insufficent people.
MLS has been developed and used since the very early time of computer ego.
Many multi-level systems which requires highly secured data protection, especially military systems, benifited from MLS.
%There are several MLS models, and due to the popularity, one is going to consider Bell-LaPadula model in the rest of this article.

A small-scale web platform for project management was created in this thesis project.
As the maturity of programming technologies (\eg Javascript in our case), given the completeness of Bell-LaPadulla (BLP) model, a relatively minor effort may result in obvious benefits.
%Since using MLS sometimes is complicated and expensive, the goal of this project is to help organizations to benefit from MLS work flow while reduce its complexity.
Only the minimal essential management functionality was being implemented into the platform, in which one will evaluate how MLS can solve real-world problems.
The main issue will be the conventional security policies applied as the access control to existing documents management systems.
%The platform divides all components into two groups, either \emph{subjects} or \emph{objects}.
%These components, then, are labeled with a variety of combinations of \emph{clearance} and \emph{classification} labels.
%The sets of labels are to be compared whenever a \emph{subject} tries to access an \emph{object} in order to decide the \emph{subject}'s permissions \eg not permitted, readable, writable etc\dots

The project outcome outlined in this thesis.
The platform shows potential to commercialization.
Even the platform only provides some limited PM features, a clear future plan has been constructed with additional features which can advance the platform to a competive product in the market emphasizing its security advantages.
%The key feature of the tool is the accessibility control of documents in an organization where many employees with various roles can work and share documents centrally; the list of advantages using MLS will be introduced apparently in this thesis paper.
Moreover, both technical and logical problems in implementing Bell-LaPadula (BLP) model into a \emph{project management tool} (PMT) were reported in this thesis, as well as their solutions if available.
Remaining limitations, and trade-offs are also recorded and explained for later improvement.

\bigskip
Keywords: Multilevel Security, Bell-LaPadula, Biba, project management, system security, security model, web development, programming project.
\endgroup			

\vfill
